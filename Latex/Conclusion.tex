\chapter[Conclusion]{Conclusion}
\section{Conclusion}

The purpose of this Master Thesis was to build a proof of concept for a novel 3D imaging system for the lymphatic system. As the reader could notice, such system relies on several computer vision techniques and various common imaging problems needed to be solved such as: camera calibration, camera pose estimation, texture mapping, 3D reconstruction and so on. Since the goal was to produce a proof of concept, only a Kinect camera coupled with an external webcam were utilized. The Kinect was used to acquire a 3D surface of the scene while the webcam was used to obtain pictures of it. Once the pictures were acquired, they could be projected at the correct locations on the 3D surface. Results showed that such concept is feasible. However, one needs to be careful about the parallax issues as discussed in Section~\ref{sec:Results} and how badly it could affect the solution in a real life scenario where the cameras will be used in a medical context.\\ 

Also, in a real world scenario, the external webcam will be replaced by an infrared camera coupled with the use of fluorochromes such as the Indocyanine Green. A feasibility study has been conducted by using the Hamamatsu PDE camera as an external camera. Results showed that the IR light emitted by the Kinect perturbed a lot the image acquisition carried out by the PDE camera (see Section~\ref{sec:Hamamatsu camera}). Future works should try to solve this issue by eventually using filters on the cameras or turning off the Kinect IR emitter when pictures are acquired by the other IR camera.\\

In conclusion, this Master Thesis built the basics for a new imaging technique. However, this 3D imaging tool is not mature enough to be used in a medical context therefore future works on it should focus on reducing the parallax and fluorescence issues.  